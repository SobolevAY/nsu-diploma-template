\chapter{Последняя глава с примерами формул и листингов}

В программе аттестации \cite{prog} ничего не сказано про оформление формул и листингов, поэтому здесь просто будут примеры и того, и другого, взятые из разных источников.

Какая-то формула:

\begin{equation}
\begin{pmatrix} 
	\dot{\varphi}\\ \dot{\theta} \\ \dot{\psi} 
\end{pmatrix} =
\begin{pmatrix}
	cos(\theta)cos(\psi) & -sin(\psi) & 0 \\
	cos(\theta)sin(\psi) & cos(\psi)  & 0 \\
	-sin(\theta)         & 0         &  1
\end{pmatrix}^{-1}
\begin{pmatrix}
	\omega_x\\ \omega_y \\ \omega_z
\end{pmatrix}.
\end{equation}

Еще какая-то формула:

\begin{equation}
\int_{a}^{b} f(x)dx = F(b) - F(a)
\label{eq:eq-sample}
\end{equation}

Здесь ссылка на формулу \ref{eq:eq-sample}.

А здесь пример листинга с подписью. Подпись оформляется аналогично подписи к картинке.

\begin{lstlisting}[language=Java,caption=Пример листинга]
List<String> list = Stream.of("item1", "item2", "item3")
						.sequential()
						.filter(new TreeSet<>()::add)
						.collect(Collectors.toList());
System.out.println(list);
\end{lstlisting}