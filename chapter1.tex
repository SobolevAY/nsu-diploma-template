\chapter{Начало содержательной части}

В основной части приводят данные, отражающие сущность, методику и основные результаты выполненной работы.

\section{Раздел начала содержательной части}

Основная часть должна содержать:
\begin{enumerate}
	\item выбор направления исследования, включающий его обоснование, методы решения задач и их сравнительную оценку, описание выбранной общей методики проведения работы;
	\item процесс теоретических и (или) экспериментальных исследований, включая определение характера и содержания теоретических исследований, методы исследований, методы расчета, обоснование необходимости проведения экспериментальных работ, принципы действия разработанных объектов, их характеристики;
	\item обобщение и оценку результатов исследований, включающих оценку полноты решения поставленной задачи и предложения по дальнейшим направлениям работ, оценку достоверности полученных результатов и их сравнение с аналогичными результатами отечественных и зарубежных работ, обоснование необходимости проведения дополнительных исследований, отрицательные результаты, приводящие к необходимости прекращения дальнейших исследований.
\end{enumerate}

\section{Второй раздел начала содержательной части}

Основная часть выпускной квалификационной работы может состоять из любого количества глав на усмотрение автора, как правило, из трех. Каждая глава содержит пункты и подпункты. Главное --- чтобы было раскрыто содержание, поставлены цели и решены и задачи. В конце каждой главы формулируются краткие выводы по результатам проведенного анализа.

\subsection{Подраздел второго раздела}

Выпускная квалификационная работа должна быть выполнена печатным способом с использованием компьютера и принтера. Текст работы выполняется на одной стороне листа одно-сортной белой бумаги формата А4 (210х297).

\subsection{Второй подраздел второго раздела}

Шрифт: Times New Roman, черный; размер основного шрифта: кегль 12-14 пт; межстрочный интервал 1,5. Поля: левое –-- 30 мм, правое –-- 10 мм, верхнее и нижнее –-- 20 мм.