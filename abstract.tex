\thispagestyle{empty}
\newpage
\setcounter{page}{2}

\begin{center}
	\textbf{АННОТАЦИЯ} \\
	\vspace{5mm}
	ВЫПУСКНАЯ КВАЛИФИКАЦИОННАЯ РАБОТА БАКАЛАВРА \\
	Наименование темы \uline{\topic\hfill} \\
	\hrulefill
\end{center}

\begin{flushleft}
	Выполнена студентом(кой) \uline{\studentformedname\hfill} \\
	\faculty, \university \\
	Кафедра \uline{\department\hfill} Группа \uline{\group} \\
	Образовательная программа \programnum\thinspace\programname \\
	\vspace{5mm}
	% Объем работы на единицу меньше реального, так как аннотация с работой не переплетается
	Объем работы: \total{page} страниц \\
	Количество иллюстраций: \totalfigures \\
	Количество таблиц: \totaltables \\
	Количество литературных источников: \total{citenum} \\
	Количество приложений: \total{attachnum} \\
	Ключевые слова: latex, шаблон, ещекакоенибудьключевоеслово
\end{flushleft}

% Текст аннотации: содержит формулировку задачи и основных результатов, их новизну и актуальность и должен отражать:
%   - объект исследования;
%   - цель работы и ее актуальность;
%   - задачи, выполненные для достижения цели;
%   - используемые методы исследования;
%   - полученные результаты и их новизну;
%   - область применения полученных результатов и рекомендации

Товарищи! Реализация намеченных плановых заданий играет важную роль в формировании форм развития. Не следует, однако, забывать, что реализация намеченных плановых заданий обеспечивает широкому кругу (специалистов) участие в формировании форм развития. С другой стороны, консультация с широким активом требует от нас анализа модели развития. Значимость этих проблем настолько очевидна, что рамки и место обучения кадров требуют определения и уточнения направлений прогрессивного развития. Идейные соображения высшего порядка, а также сложившаяся структура организации играют важную роль в формировании систем массового участия.

Значимость этих проблем настолько очевидна, что дальнейшее развитие различных форм деятельности в значительной степени обуславливает создание позиций, занимаемых участниками в отношении поставленных задач. Не следует, однако, забывать, что начало повседневной работы по формированию позиции способствует подготовке и реализации системы обучения кадров, соответствует насущным потребностям.

\begin{flushleft}
	\vfill
	\student, \\ % Место для подписи
	\today
\end{flushleft}