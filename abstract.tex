\thispagestyle{empty}
\newpage
\setcounter{page}{2}

\begin{center}
	\textbf{АННОТАЦИЯ} \\
	\vspace{5mm}
	ВЫПУСКНАЯ КВАЛИФИКАЦИОННАЯ РАБОТА БАКАЛАВРА \\
	Наименование темы \uline{\topic\hfill} \\
	\hrulefill
\end{center}

\begin{flushleft}
	Выполнена студентом(кой) \uline{\studentformedname\hfill} \\
	\faculty, \university \\
	Кафедра \uline{\department\hfill} Группа \uline{\group} \\
	Образовательная программа \programnum\thinspace\programname \\
	\vspace{5mm}
	% Объем работы на единицу меньше реального, так как аннотация с работой не переплетается
	Объем работы: \total{page} страниц \\
	Количество иллюстраций: \totalfigures \\
	Количество таблиц: \totaltables \\
	Количество литературных источников: \total{citenum} \\
	Количество приложений: \total{attachnum} \\
	Ключевые слова: latex, шаблон, ещекакоенибудьключевоеслово
\end{flushleft}

Аннотация выпускной квалификационной работы должна содержать: название работы, сведения об объеме (количестве страниц), количестве иллюстраций и таблиц, количестве использованных источников, количестве приложений, перечень ключевых слов; текст аннотации (содержит формулировку задачи и основных результатов, их новизну и актуальность). Ключевые слова в совокупности дают представление о содержании.

Текст аннотации должен отражать:
\begin{enumerate}
	\item объект исследования;
	\item цель работы и ее актуальность;
	\item метод исследования;
	\item полученные результаты и их новизну;
	\item область применения и рекомендации.
\end{enumerate}

Требования к содержанию, построению и оформлению текста аннотации определяются ГОСТ 7.9-95 «СИБИД. Реферат и аннотация. Общие требования». Объем аннотации не должен превышать 1 страницы.

\begin{flushleft}
	\vfill
	\student, \\ % Место для подписи
	\today
\end{flushleft}